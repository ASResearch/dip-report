\section{性质分析}
\begin{frame}\frametitle{作弊行为}
	相比黄皮书,作弊行为更复杂:
\begin{itemize}
\item 收买用户:开发者通过各种手段例如贿赂用户从而获得用户的全部投票权;
\item 恶意分拆:开发者将自己的DApp强行分拆成若干个DApp以获得所有分拆的 DApp的总奖励;
\item 女巫攻击:创建大量虚假地址来获取更高投票效用(不同于NR的女巫攻击)。
\end{itemize}
\end{frame}

\begin{frame}\frametitle{抗作弊特性}
抗收买:
\begin{property}
在DIP模型中,对于一个利益最大化的用户,一般而言,他会将手中的票投给多个不同的DApp。
\end{property}

\begin{corollary}
被收买的用户对DApp排名分的总贡献远远小于正常的用户。
\end{corollary}
\end{frame}

\begin{frame}\frametitle{抗作弊特性}
抗恶意拆分:
\begin{property}
	在所有投票者均为正常用户的情况下,开发者通过DApp分拆不会提升他的收益。
\end{property}

\begin{corollary}
	即使引入被收买者的情况下,开发者进行DApp恶意分拆不会提升他的收益。
\end{corollary}
\end{frame}

\begin{frame}\frametitle{抗作弊特性}
抗女巫攻击:
黄皮书已经保证大量创建子账户无法提升NR值,进一步地

\begin{property}
	\label{p3}
	对于任何用户,进行女巫攻击不会增加他所投的DApp的排名分以及最终奖励。
\end{property}
\end{frame}


\section{实现和未来工作}
\begin{frame}\frametitle{实现和未来工作}
DIP具体实现:
\begin{itemize}
\item 发放奖励:设置专用的奖励发放账户$D$,增加发放奖励的交易类型,\texttt{dip}交易;
\item 协议更新:基于NF,更新区块结构(LLVM IR);
\end{itemize}
\end{frame}

\begin{frame}\frametitle{实现和未来工作}
未来工作:
\begin{itemize}
\item 多维投票行为:解析智能合约调用,引入资金交易这个信息;
\item DApp间调用:考虑合约内部调用,进一步传递投票效用;
\end{itemize}
\end{frame}