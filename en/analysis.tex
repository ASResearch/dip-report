\section{Property Analysis}
\label{section:properties}
Have introduced the Developer Incentive Protocol, in this section we analyze  manipulations that could occur in practice and the properties against manipulation of  DIP. On the voters and developers' points of view respectively, manipulations including buying over, maliciously splitting DApp, Sybil attack and so on.
\subsection{Buy Over Voters}
The so-called buy over means that a developer lets voters' cast all their votes to the developer's DApp, by the means of bribing or other, which is vary common in the real life. Here we suppose that all voters are self-interest. We assume that normal voters only care about the ranks of the DApps they like, rather than the final reward of developers. In other words, a normal voter want to maximize the total
weighted ranking score of all DApps he likes. Our quadratic ranking algorithm guarantees the following property:
\begin{property}
	\label{p1}
	In the DIP model, for a self-interested normal voter, generally, he will case his votes to multiple DApps.  
\end{property}
We illustrate the property by the following model: suppose the weights that voter $a_i$ values all DApps are $b_{i1}, b_{i2}, ..., b_{in}$ respectively (can be regarded as the true preference of the voter to all DApps). Taking the form of~\ref{eq:sqrt}, the voter's contributory values satisfy 
$$\frac{b_{i1}}{\sqrt{\nr_{i1}}}=\frac{b_{i2}}{\sqrt{\nr_{i2}}}=\cdots=\frac{b_{in}}{\sqrt{\nr_{in}}}$$,
in other words, voter $a_i$'s contributory values matches his true preference to DApps. The detailed proof is in Section~\ref{subsection:proof1}.

Transitional voting models usually compute the ranking score linearly, i.e.,
$$g(\nr_{1j},\nr_{2j},...,\nr_{mj}) = \sum_{i=1}^m \nr_{ij}$$.
In this model, a rational voter only cast his votes to the DApp he likes the most. In comparison, formula ~\ref{eq:sqrt} can promote the interactions between voters and DApps, due to the property of the square root function. In other words, voters voting for multiple DApps maximizes the utilization of his voting capacity. Similar analysis can be found in~\ref{buterin2018liberal}. To sum up, a voter would vote for multiple DApps and keeps the priority of DApps he likes the most simultaneously, i.e. the ratio equation above.

In practice, sometimes transitional  linear voting model will limits the maximum votes for a voter to a DApp, to forcibly let voters disperse their votes, while our algorithm achieve the same goal by the means of essentially incentive,  with a more elegant and simple mathematical expression. 

\begin{corollary}
	The total contributory values of a  voter who is bought over is far less than the total contributory values of a normal voter.
\end{corollary}
For a voter $a_i$ who is bought over, he can at most offer the DApp with contributory value with amount $\sqrt{\nr_i}$. For a normal voter who is not bought over, assume that he plans to vote for $K$ DApps,\footnote{$K$ reflects the number of DApps of which the contributory values from the voter can discriminate with other DApps, which is usually larger than 1, as long as the voter's value weights of the $K$ DApps are not extremely distributed, i.e., the voter only votes for a particular DApp and the votes for other DApps tent to 0} when  his value weights to thees DApps,are uniformly distributed, the total increment of ranking scores  over all DApps caused by the voter is at the over of $O(\sqrt{K\nr_i})$, that is, the efficiency of a normal voter is $K$ times the efficiency of a voter who is bought over. Therefore the cost of the manipulation about buying over is increased.

\subsection{Malicious Splitting}
\label{subsec:5.2}
For developers, another manipulation is to maliciously split their DApps, in order to obtain the total rewards of all split DApps. Intuitively, splitting can increase the number of DApps that participant the reward mechanism and thus increase the total final rewards. However our model guarantees that it is not the case. Here we assume that all developers concern their final rewards (total bonus) as well as the implicit utility caused by the improvement of ranks.

Specifically, as the algorithm of final rewards, the convexity of formula~\ref{eq:distribution} grantees the following property
\begin{property}
	\label{p2}
    If all voters are normal, splitting DApps will not increase the reward of the developer. 
\end{property}
It is assume that a normal voter belongs to the following to cases: i) voter simply distributes his votes that are supposed for the original DApp to split DApps. Such case occurs when an application has different smart contract address for invocation. ii) Suppose that the voter values the original DApp with weight $a$, and with weights $b$ and $c$ for two split DApps respectively, then $c>a+b$. Such case can be illustrated that after splitting, the split DApps' qualities are greatly decreased and lack of linkages. So the total qualities of split DApps' is lower than the original DApp. In both cases, the final reward of the developer does not increase. Detailed proof  is in ~\ref{subsection:proof2}.

Further more, a developer can buyer over voter and split his DApp simultaneously: he first splits his DApp to $K$ DApps. Then, he lets his bought-over voters distribute their votes uniformly to split DApps, thus  maximizing the utilization of  bought-over  voters. We have the following corollary against this case:
\begin{corollary}
	\label{c1}
	Even with the introducing of bought-over voters, the developer can not increase his final rewards by splitting his DApps.
\end{corollary}
\subsection{Sybil Attack}
Detailed proof is in Section~\ref{subsection:proof3}.

It is notable that, the rank of DApps will be decreased if developers split DApps, thus implicit utility is also decreased. In summary, our algorithm essentially prevents malicious splitting. 

Without doubt that is a developer developers multiple different  DApps, since there is no mirrored of split relations amount the DApps, the utility of the developer is not affected.

\subsection{Sybil Attack}
By generalized Sybil Attack we mean an attack subverts the reputation system by creating a large number of pseudonymous identities, using them to gain a disproportionately large influence~\cite{quercia2010sybil}. In Nebulas Yellow Paper~\cite{Nebulasyellowpaper}, the properties of NR against manipulations that increase NR by creating a large number of new accounts have been proved. So in the DIP ranking algorithm, voters are also not able to increase NR by creating new accounts, i.e.,
$$\mathcal{C}(c)>\mathcal{C}(a)+\mathcal{C}(b)$$
where $c$ is the original account, $a,b$ are the split sub-accounts. According to formula~\ref{eq:vote_rate} their voting capacities satisfy the following constraint: 

\begin{align}
	\label{eq:sqrt_nr}
	\sqrt{\nr_{a+b}}>\sqrt{\nr_a}+\sqrt{\nr_b}
\end{align}
Suppose that the propose of a voter for Sybil attack is to increase the ranking score of a specific and the final rewards of its developer. According to the constraint above we have the following property:

\begin{property}
	\label{p3}
    For any voter, executing Sybil attack will not increase the ranking score of the DApp he votes and the final reward of the DApp's developer. 	
\end{property}
So the property against Sybil attack is guaranteed.


