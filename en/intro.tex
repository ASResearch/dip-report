\section{Introduction}

Generally, developers develop applications on some application platforms (like
Windows\footnote{\url{https://www.microsoft.com/en-us/windows}}, Linux\footnote{\url{https://en.wikipedia.org/wiki/Linux}},
macOS\footnote{\url{https://en.wikipedia.org/wiki/MacOS}},
iOS\footnote{\url{https://en.wikipedia.org/wiki/IOS}},
Android\footnote{\url{https://en.wikipedia.org/wiki/Android}} \etc) and
benefit from their applications in traditional software development industry.
The way to get the benefits varies for different developers, including but not
limited to salaries paid by software enterprises, revenue by selling the
application licenses or displaying advertises in their applications.

However, the enterprises who build the application platforms also benefit
from the applications, while the benefits are not shared with the developers.
Let's take the operating system as an example here: a UI/UX designer wants to use \texttt{Sketch},
as we know that the application only works on {macOS} device, so besides
paying for the application itself, the designer needs to pay Apple\footnote{\url{https://en.wikipedia.org/wiki/Apple_Inc.}}
for the device  to use the application. Apparently, Apple benefits from such user while
Apple does not share the benefits with the \texttt{Sketch} developers.
Another similar example is that users have to pay Apple or
Microsoft\footnote{\url{https://en.wikipedia.org/wiki/Microsoft}} to use
AutoCAD\footnote{\url{https://en.wikipedia.org/wiki/AutoCAD}}. In such cases,
the key factor that users choose a platform is whether the platform
supports required applications for users. In other words, high-quality
applications are critical to the development of an application platform. Based on the above considerations,  application platforms ignore the interests of developers, to a certain extent, infringing the interests of developers.

In the blockchain industry, the interests of \dapp(Decentralized Application) developers are ignored by platforms  as  well.
 In 2004, Ethereum community proposed ``Smart Contract'',
which extended blockchains' ability from peer-to-peer
cryptocurrency networks to decentralized application platforms. However, in comparison with traditional centralized development industry, the ways of obtaining revenue for developers have no significant difference --- decentralized application developers still can not benefit
 from the increment of the blockchain system's value.


Generally speaking,  new-block rewards represent incremental values of the blockchain system and the distribution of such rewards determines the incentive direction of the decentralize system. In our opinion, a blockchain system's incremental value essentially comes from the implicit values from users' data, which should be distributed to all contributed parties, including \dapp deveploers. However, what we see in practice is that, in most PoW blockchain systems, represented by Bitcoin, new-block rewards are distributed to miner nodes; In PoS (proof of stake) based blockchain systems, new-blocks rewards are assigned to  stake holders. Along with it, the interests of \dapp developers are somewhat infringed.


Conceptually, a \dapp is a set of smart contracts with a series of specific
functionalities, while a smart contract is a computer protocol intended to
digitally facilitate, verify, or enforce the negotiation or performance of a
contract. Smart contracts allow the performance of credible transactions
without third
parties\footnote{\url{https://en.wikipedia.org/wiki/Smart\_contract}}.

From the technical architecture's point of view, most \dapp usually uses smart contracts as the back-end, while using common front-end technologies and its interactions. {\dapp}s' forms can be either a traditional PC client, mobile applications or web applications.

We believe that the relationship among decentralized application platforms, \dapp developers and \dapp users is mutually reinforcing and symbiotic. Firstly, the emergence of decentralized application platforms enlarges the group of blockchain developers. More and more developers try to develop {\dapp}s that meet different requirements and benefit from the development of {\dapp}s. Secondly, \dapp developers provide a rich variety of {\dapp}s, expanding the application scenarios of the blockchain, and bringing more incremental users to the blockchain. Finally, \dapp users drive the continuous optimization and upgrade of decentralized application platforms, increasing the mobility of tokens on the decentralized application platform, making the whole blockchain system develop.

It should be noted that the developers described here only refer to developers on the decentralized application platform, not specifically the Nebulas developers, nor the developers of the blockchain system itself.
Notice that we mean \dapp developers instead of Nebulas \dapp developers or
blockchain system developers. Thus, we shall use developers short for \dapp
developers without ambiguity. Also, a \dapp developer may be a stake holder.
He may benefit from being a stake holder and that benefit is not considered
as sharing the increase value of blockchains. The interests of being a developer can still be ignored or infringed.


It's inappropriate to directly distribute a platform's increased value to
corresponding application developers. On one hand, the revenue is owned by
centralized organizations, like an enterprise, and application developers have
no chance to know the details of or participate in the sharing of revenues. Second, it
is difficult to quantify each developer's contribution to the growth of
application platforms so the fairness of the reward mechanism is hard to be guaranteed. Fortunately, this situation can be changed in blockchain
industry since each invocation to smart contracts by each user is
publicly recorded on blockchain. Thus, it is possible to \emph{reward or incentive each \dapp developer by
quantifying each \dapp's contribution}.

An ideal incentive mechanism should satisfy some basic properties:
\begin{itemize}

\item Fairness: the protocol should maintain objectivity when rewarding developers, that is, every {\dapp}s should be equally treated and their usages are evaluated veritably, even there are some potential manipulations.

\item Effectiveness: the reward should reflect user preference, that is, the {\dapp}s with high reward are ones that frequented by active users while the {\dapp}s with low or no reward are unwelcome.
\end{itemize}


In this paper, we propose Developer Incentive Protocol (DIP), which aims at rewarding and incentivizing  developers, enabling the developers to benefit from the development of the decentralized application platform. Naturally, an ideal developer incentive protocol does not exist since the users' evaluations to  {\dapp}s are subjective and multi-dimensioned. So the DIP introduced in the paper still has space for improvement. However, the balance in this mauve paper is innovative, that is, under the premise of guaranteeing the interests of \dapp developers, in terms of resistance to manipulation, we make the greatest efforts.

DIP is designed based on the existing Nebulas Rank
(NR)~\cite{Nebulasyellowpaper} and it benefits from some good features of NR\@.
Intuitively, DApp evaluation is reduced to a voting process in DIP\@.

An invocation from a certain user is treated as a vote and user's voting
capacity is a function of his/her NR\@. The developers will get the rewards from the system eventually, according to the voting results.

Besides giving the theoretical model of Developer Incentive Protocol, we also
analyze the properties against manipulations and illustrate the
implementation of DIP, such as, how to adjust and update DIP\@, which specify the direction of the actual landing of the DIP.

\whitepaper{
Special Hint: As the mauve paper focus on discussing Developer Incentive Protocol, it greatly upgrades and expands the relevant chapters of the Nebulas Technology White Paper~\cite{Nebulaswhitepaper}(version 1.02 released in April 2018). Compared with the conceptual demonstration one year ago, after a year of in-depth thinking and practical verification, we are confident and able to design more rigorous algorithms and provide clear solutions or directions for more practical details of the Nebulas Incentive Protocol.
}
