\section{Introduction}

Generally, developers develop applications on some application platforms (like
Windows\footnote{\url{https://www.microsoft.com/en-us/windows}}, Linux\footnote{\url{https://en.wikipedia.org/wiki/Linux}},
macOS\footnote{\url{https://en.wikipedia.org/wiki/MacOS}},
iOS\footnote{\url{https://en.wikipedia.org/wiki/IOS}},
Android\footnote{\url{https://en.wikipedia.org/wiki/Android}} \etc) and
benefit from their applications in traditional software development industry.
The way to get the benefits varies for different developers, including but not
limited to salaries paid by software enterprises, revenue by selling the
application licenses or displaying advertises in their applications.

However, the enterprises who build the application platforms also benefit
from the applications, while the benefits are not shared with the developers.
Let's take the operating system as an example here: a UI/UX designer wants to use \texttt{Sketch},
as we know that the application only works on \texttt{macOS} device, so besides
paying for the application itself, the designer needs to pay Apple\footnote{\url{https://en.wikipedia.org/wiki/Apple_Inc.}}
for the device  to use the application. Apparently, Apple benefits from such user while
Apple does not share the benefits with the \texttt{Sketch} developers.
Another similar example is that users have to pay Apple or
Microsoft\footnote{\url{https://en.wikipedia.org/wiki/Microsoft}} to use
AutoCAD\footnote{\url{https://en.wikipedia.org/wiki/AutoCAD}}. In such cases,
the key factor that users choose a platform is whether the platform
supports required applications for users. In other words, high-quality
applications are critical to the development of an application platform. Based on the above considerations,  application platforms ignore the interests of developers, to a certain extent, infringing the interests of developers.

In the blockchain industry, the interests of
DApp (Decentralized Application) developers are ignored by platforms  as  well.
 In 2004, Ethereum community proposed ``Smart Contract'',
which extended blockchains' ability from peer-to-peer
cryptocurrency networks to decentralized application platforms. However, in comparison with traditional centralized development industry, the ways of obtaining revenue for developers have no significant difference -- decentralized application developers still can not benefit
 from the increment of the blockchain system's value.

Generally speaking,  new-block rewards represent incremental values of the blockchain system and the distribution of such rewards determines the incentive direction of the decentralize system. In our opinion, a blockchain system's incremental value essentially comes from the implicit values from users' data, which should be distributed to all contributed parties, including DApp deveploers. However, what we see in practice is that, in most PoW blockchain systems, represented by Bitcoin, new-block rewards are distributed to miner nodes; In PoS (proof of stake) based blockchain systems, new-blocks rewards are assigned to  stake holders. Along with it, the interests of DApp developers are somewhat infringed.

Conceptually, a DApp is a set of smart contracts with specific
functionality, while a smart contract is a computer protocol intended to
digitally facilitate, verify, or enforce the negotiation or performance of a
contract. Smart contracts allow the performance of credible transactions
without third
parties\footnote{\url{https://en.wikipedia.org/wiki/Smart\_contract}}.
Currently, most DApps use smart contracts as backend, and applications on PC,
mobile or web as frontend.

We believe that blockchain systems or DApp platforms, DApp developers and DApp
users are closely related with each other. First, DApp platforms provide
necessary facilities for DApp developers so that more and more DApp developers
can build their own DApps and benefit from the DApps. Second, DApp
developers build DApps for different scenarios, which bring more and more
users to the DApp platforms. Finally, DApp users produce onchain data,
increase liquidity, and increase the value of a blockchain system.

Notice that we mean DApp developers instead of Nebulas DApp developers or
blockchain system developers. Thus, we shall use developers short for DApp
developers without ambiguity. Also, a DApp developer may be a stake holder.
He/she may benefit from being a stake holder and that benefit is not considered
as sharing the increase value of blockchains.


It's inappropriate to directly distribute a platform's increased value to
corresponding application developers. First, the revenue is owned by
centralized organizations, like an enterprise, and application developers have
no chance to know the details or participate in sharing the revenue. Second, it
is difficult to quantify each developer's contribution to the growth of
application platforms. Fortunately, this situation can be changed in blockchain
industry since each invocation for smart contracts by each user is
publicly recorded on
blockchain. Thus, it is possible to \emph{reward or incentive each DApp developer by
quantifying each DApp's contribution}.

An ideal incentive mechanism should satisfy some basic properties:
\begin{itemize}
\item Faireness: the protocol should maintain objectivity in the process of rewarding developers, that is, every DApp should be evaluated equally even there are some potential manipulations.
\item Effectiveness: the reward should reflect user preference, that is, the DApps with high reward are ones that frequented by active users while the DApps with low or no reward are unwelcome.
\end{itemize}

In this paper, we propose Developer Incentive Protocol (DIP), which aims at rewarding developers and makes the developers to benefit from the process in a decentralized application platform. Naturally, the ideal developer incentive protocol is not exist since the user satisfaction evaluations to DApps are subjective and multi-dimensioned. And the DIP introduced in the paper may have some potential weaknesses. However, the research is still creative and has a balance of anti-manipulation and developers' rights. We give our best to improve the performance of anti-manipulation, while still keeping the developers rights.

DIP is designed based on the existing Nebulas Rank
(NR)~\cite{Nebulasyellowpaper} and it benefits from some good features of NR\@.
Intuitively, the DApp evaluation of users is reduced to voting problem in DIP\@.
An invocation from a certain user is treated as a vote and user's voting
capacity is a function of her NR\@. The developers will get the rewards from system eventually, corresponding to the voting result.

Beyond giving the theoretical model of Developer Incentive Protocol, we also
have analysis about the properties of anti-manipulations and illustrate the
implementation of DIP, such as, how to adjust and update DIP\@. These works
above will provide the direction of practical application of DIP\@.

\whitepaper{
Special Hint: The mauve paper is devoted to Developer Incentive Protocol which is different from the description in our whitepaper (version 1.02 released on April 2018) [3] with numerous new upgrades and extensions. This is because we keep thinking and verifying the algorithm in our whitepaper. And now we are more confident and capable to make it more rigorous. We use a different format (like this paragraph) to emphasize the relevant updates presented in this mauve paper.
}
