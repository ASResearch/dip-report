\section{Implementation of DIP}

\noindent
The complete implementation of DIP is out of scope of this paper. So here we only discuss the key issues should be handled when implementing DIP\@.

\subsection{How to Distribute Rewards}
\noindent
For distributing rewards, a special account $D$ should be established. In the meanwhile, a part of the new-block rewards is transferred to $D$ according to a fixed ratio.


Developers' deserved rewards will be sent regularly.\footnote{The time interval of sending rewards equals to the sampling interval in Section~\ref{subsection:interval}}. In order to send rewards on the blockchain, the private key of the account that sends rewards is required for signing the transaction of sending. Therefore, for the purpose of security, the account $D$ for sending rewards needs a special treatment.

Firstly, a special kind of transaction is added to the system, denoted by \texttt{dip} transaction, which contains the information about the amount of deserved reward of a developer and the height of the blockchain. Secondly, the system refuses all transactions other than \texttt{dip} initiated by $D$, to ensure that no account can extract tokens from $D$. Finally, verification nodes on the blockchain  need to verify the \texttt{dip} transactions. Particularly, verification nodes need to run DIP locally and verify whether the data in \texttt{dip} transactions coincident with local results.

Through the methods mentioned above, not only the rewards for developers are distributed normally, but also the security of account $D$ for sending rewards is ensured.

\subsection{Updating of DIP}
\noindent
As we know, the DIP is closely related to the whole ecosystem. As the variance of the ecosystem, DIP ought to be updated, Specifically, the parameters in the DIP model can be updated. So how to effectively update the DIP turns out to be a key issue.

For this, we use Nebulas Force to iteratively update DIP\@.

We update the blocks' structures to contain algorithms and parameters in DIP (by the form of LLVM IR). Nebulas Virtual Machine (NVM), as the engine of the algorithm, gets algorithm and parameters from blocks and run algorithms, to obtain the amount of tokens that an account deserves.

When algorithms and parameters need to be updated, Nebulas group will work together with the community, let new blocks contain updated algorithms and parameters, to ensure the timeliness and smoothness of the whole updating process and avoid any possible forks.
