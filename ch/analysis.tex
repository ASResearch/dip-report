\section{性质分析}
这章我们主要分析我们的排名算法所剧本的各种防作弊性质。主要包括防收买,防恶意分拆DApp,防女巫攻击等等。在实际运作中,我们可能根据实际情况对具体公式会有所调整,但这些性质仍能够保证大致满足。

\subsection{抗收买}
所谓收买是指DApp开发者通过各种手段获得用户的全部投票权。这里我们假设所有正常用户都是利益最大化的。我们认为用户关心的是他心目中DApp在排行榜上所处的排名,而不关心DApp开发者最终会获得多少奖金。亦即,每个用户乐于最大限度的提升自己心目中优秀DApp的排名分(score值)。我们的二阶排名算法保证了下面这个特征:

\begin{property}
	\label{p1}
	对于一个利益最大化的用户,一般而言,他会将手中的票投给多个不同的DApp。
\end{property}
我们用如下模型来具体说明:

不失一般性,假设用户$a_i$所有DApp的价值权重为$b_{i1},b_{i2},...,b_{in}$(可理解为用户的打分),则该用户最终分配的分贡献值满足
$$\frac{b_{i1}^2}{\nr_{i1}}=\frac{b_{i2}^2}{\nr_{i2}}=\cdots=\frac{b_{in}^2}{\nr_{in}}$$
具体证明见附录。

若使用更常见的线性排名分算法(即$score$为所有的分贡献值之和,而不是根号之和),理性用户则只会将所有票投给自己最喜爱的DApp。相比之下,我们的算法更能促进用户和DApp之间的交互。造成这样的原因在于根号函数的特性,即用户把票投给多个DApp相当于他的总票数变多了。故用户会倾向于投给多个DApp的同时保证自己最喜欢DApp的领先性,即上面的比例等式。

实际生活中,某些常见投票方式为限制用户为单个目标投票的最大票数,从而强制让用户投给多个目标。而我们的算法则是通过激励从本质上达到同样的目的,且在数学表达更为简洁,更适合作为激励协议的内容。

下面这个推论体现了我们的算法抗收买的特性:
\begin{corollary}
被收买的用户对DApp排名分的总贡献远远小于正常的用户。
\end{corollary}
显而易见,已经被收买的用户$a_i$最多给收买其的DApp开发者贡献$\sqrt{\nr_i}$。而一个正常用户,假设他认为好的DApp有$K$个,在其对这些DApp的价值权重是分布较为均匀的情况下,他给所有DApp带来的总排名分提升大约在$O(\sqrt{K})$这个级别\footnote{$K$反应的是该用户给出具有区分度的投票的DApp数目,通常是大于1的,只要该用户对DApp价值权重分布不是太极端,即只喜欢某个特定的DApp而对其他DApp打分都趋近于0。},即他起到的作用是被收买用户的$O(\sqrt{K})$倍。这样就一定程度上提高了开发者收买用户的代价。

\subsection{抗恶意分拆}
(增加开发者开发多个正常DApp的讨论?)

所谓恶意分拆是指DApp开发者将自己的DApp强行分拆成两个或多个低质量的DApp以获得所有分拆的DApp的总奖励。我们认为DApp开发者关心的是最终获得的总奖金。同时也有一定的高排名带来隐性收益。

我们的最终奖励的凹函数性能保证下面的特征
\begin{property}
	在所有投票者全是正常用户的情况下,开发者进行DApp分拆不会提升他的收益。
\end{property}
这里假设正常用户存在下列两种情形。其一用户面对DApp分拆选择简单的把原本投给分拆前DAPP的票分散开来(分给拆分后的DApp)。其二,假设拆分之前用户对该DApp的价值权重为$c$,对两个拆分之后的DApp的价值权重分别为$a$和$b$,则$c \geq a+b$。这个可以理解为DApp拆分之后两者的质量将大幅下降且缺乏联动性,导致两者的质量和比原DApp还要差。该特征的具体证明见附录。

开发者还可采取的作弊手段为同时进行DApp分拆和收买,例如,先分拆成$K$个DApp,然后让被收买的用户均匀的将票投在自己拆分的$K$个DApp上以实现效用最大化。我们有下面的推论:
\begin{corollary}
	即使引入被收买者的情况下,开发者进行DApp分拆不会提升他的收益。
\end{corollary}
具体证明见附录。

值得注意的是,开发者将DApp进行拆分同时也会降低在排行榜上的排名,从而减少排名带来的隐形收益。综上所述,我们的算法能从本质上防止DApp分拆的进攻手段。

\subsection{抗女巫攻击}
所谓女巫攻击是指一个原本高NR的用户将自己的财产转移到多个新账户里,然后多个账户作为被收买的新用户统一给某个DApp投票。

解决这个问题的主要手段是利用NR本身的抗作弊性质。假设进行女巫攻击的用户原NR值为$\nr_{a+b}$, 进行财产拆分之后两个新用户的NR值分别为$\nr_a$与$\nr_b$。由于我们只需要考虑该用户给同一个DApp投票,故我们只需比较$\sqrt{\nr_{a+b}}$与$\sqrt{\nr_a}+\sqrt{\nr_b}$的大小。若能满足
\begin{align}
\label{sybil}
\sqrt{\nr_{a+b}}>\sqrt{\nr_a}+\sqrt{\nr_b}
\end{align}
则以此类推可得到将NR值拆分成多个用户投票效用也不会增加。

我们可通过下面三种方案来实现这个目标:

\begin{itemize}
	\item 通过调整NR本身的参数使得在一定范围内满足(\ref{sybil})。虽然在$\nr_a,\nr_b$均趋于无穷时有$\nr_{a+b}=\nr_a+\nr_b$,与(\ref{sybil})矛盾,但因为用于女巫攻击的新用户NR值都较低,故我们只需在$\nr_a,\nr_b$时都很低时(\ref{sybil})成立即可,而这是NR通过调参所能达到的。
	
	\item 通过对所有参与投票的用户的NR值进行预处理,一个简单的方案为将所有NR值直接平方。那么(\ref{sybil})可根据黄皮书\cite{Nabulasyellowpaper}里Wilbur函数的性质$f(x_1+x_2)>f(x_1)+f(x_2)$直接得到。值得一提的是,将NR进行平方的预处理仅仅用在DIP中,而对NR的其他应用均不作处理,进而也不会损害NR本身的性质。
	
	\item 通过只取NR排名前$L$或$L\%$的用户。这样拆分成多个小NR用户可能无法达到排名要求从而增加女巫攻击的代价。而这也是白皮书\cite{Nabulaswhitepaper}里所采取的方法。
\end{itemize}
我们认为,通过上述三种方案的结合使用能够有效防止女巫攻击。