\section{开发者激励协议的实现}
开发者激励协议的完整实现不在本文的讨论范围内,此处仅讨论系统实现中所需要解决的一些关键问题。

\subsection{如何发放奖励?}
为了发放奖励,需要一个专用的奖励发放账户$D$,同时,每个出块的奖励按照固定的比例发送到$D$。

开发者所应得的奖励会定期发送\footnote{奖励发送时间间隔等于~\ref{subsection:interval}小节中的采样时间。},为了在链上发送相应的奖励,需要发出账户的私钥对发出的交易进行签名,因此,为了奖励发出账户$D$的安全,需要对该账户进行特殊处理。

首先,系统需要增加一个用于发放奖励的交易类型,\texttt{dip}交易,\texttt{dip}交易中包含了一个DApp开发者账户获得奖励的数量及所在区块高度等信息。其次,
系统将拒绝$D$发出的除\texttt{dip}交易之外的一切交易,以保证无人能从$D$中提出用于奖励的代币。最后,区块链系统中的验证节点需要对\texttt{dip}交易进行验证,
具体来说,验证节点需要在本地重新执行开发者激励协议,并验证\texttt{dip}交易中的数据是否与本地的计算结果一致。

通过上述方式,既能保证开发者奖励的正常发放,又能保证奖励发出账户$D$的账户安全。

\subsection{开发者激励协议的更新}
我们深知,开发者激励协议是和整个生态紧密相关的,随着生态的不断变化,开发者激励协议的计算也需要不断更新,尤其是其中的各个参数。如何快速地更新开发者激励协议的计算非常关键。
对此,我们将通过星云原力(Nebulas Force)来保障核心星云指数计算的更新迭代。

我们会更新区块结构,新的区块结构中将包含开发者激励协议的算法及参数(以 LLVM IR 形式),星云虚拟机(NVM)作为算法的执行引擎,从区块中获得开发者激励协议的算法及参数,并执行算法,在节点内获得账户的奖励代币数量。

在算法或参数需要更新时,我们将和社区一起协作,让新的区块中包含入最新的算法及参数,从而保证整个更新过程的及时性及平滑性,亦避免了可能到来的分叉。
