\section{概要}

%1. 区块链关于DApp的介绍 

2014年以太坊的出现,标志着区块链行业进入2.0时代\cite{buterin2014next},其中支持图灵完备的智能合约的引入使得区块链应用开发者高效快地开发上层应用变为可能。具体而言,一个分布式应用(DApp)可以理解为为了实现特定功能的一系列智能合约的集合。正如广为人知的AppStore上的应用一样,DApp可以包括游戏、博彩、社交等多种类型。相比于传统的App应用,DApp具有如下特性:

\begin{itemize}
	\item 应用数据分布式存储在区块链上所有节点,而不依赖于任何中心化的服务器,数据公开透明、不可篡改;
	\item 应用逻辑完全由智能合约代码决定,绝大多数智能合约会公布其源代码,从而保证可信度
\end{itemize}

对于区块链平台而言,DApp应用数量和质量直观反映了其社区繁荣程度,也影响着平台发展潜力。迄今为止,根据DAppradar统计以太坊上已有超过632款DApp\footnote{https://DAppradar.com/DApps.},相比之下目前大部分其他基础公链的DApp发行数量都远不及以太坊。值得注意的是,星云链一直致力于发展DApp生态并为开发者提供友好的开发平台。经过历时两个多月的激励计划,目前星云链上已有超过6871个DApp\footnote{https://incentive.nebulas.io/summary.html.}。
%同时,星云和eos上也分别有xxx和xxx DApp。

伴随着DApp爆炸式增长,暴露出的问题是DApp的质量参差不齐。因此,如何对海量的DApp进行评价以提供更好的用户体验,以及如何促进激励开发者开发新的高质量DApp成为每条公链必须解决的问题。

%2. 引入开发者激励协议的必要性(何谓激励?)

所谓激励是指通过特定的方法让用户产生行为动机。对应地,\textcolor{red}{星云开发者激励协议(DIP)是指以鼓励开发者开发优秀DApp为目的,以一套具有抗作弊性的DApp排名算法为核心,以对优秀DApp开发者给与NAS奖励为手段的一系列开源的激励机制的集合。}简单而言,优秀的DApp将在我们的排名算法中获得较高排名进而获得较高的奖励(在星云链上以官方代币NAS形式发放),以此来激励开发者。 我们期望,DIP能使每一个理性的诚实DApp开发者自发的设计优秀的DApp,正如日常生活常见的评奖活动一样。同时,每一个理性的非诚实的开发者无法通过各种作弊手段来骗取NAS奖励。


我们认为开发者激励协议需要满足如下基本性质:
\begin{itemize}
	\item 公开性:链上的DApp激励协议与传统的评奖方式最大的不同在于,所有评分的机制必须是完全公开的,且其中任何统计,计算,评选的过程都是全程可见的。这样就杜绝了传统中心化评奖暗箱操作的可能。同时也不会出现票数统计出错等情况。最后,根据评选结果分配奖励的过程也会保证被执行,奖励分配正如链上交易一样可被追溯。
	\item 有效性:这也是任何评选机制所要满足的基本性质。我们期望DApp评分能够真实反映用户的评价,即排名高的DApp是活跃用户所喜欢的且经常被调用的,而评分低的DApp是用户鲜有问津的。
	\item 抗作弊:对于任何排名算法,都需要解决各类作弊问题。对于DIP而言,主要存在两类作弊问题。1.女巫攻击:区块链技术的一个重大特点就是一个用户建立新的节点地址代价是很小的。所以一个用户有可能建立多个由他控制的地址,并将他们伪装成多个正常用户来参与评选。一个好的激励协议应当保证每个用户无法通过女巫攻击带来巨大额外收益。2.收买:由于我们衡量DApp好坏的主要指标是活跃用户调用的次数,一个DApp开发者有可能收买大量用户让他们调用自己的DApp以提高自己的排名从而获得更多奖励。这种作弊方式原则上无法杜绝,但我们期望激励协议能够让此类收买需要付出的代价变得很高以减少其出现的概率。
\end{itemize}

事实上,正如目前区块链中存在不可能三角(The Imposible Triangle)一样\footnote{https://news.8btc.com/the-impossible-triangle-of-blockchain.},设计一个能够满足上述特性的激励协议存在很大挑战,同时我们需要保证一定的高效性与可扩展性。基于已有的星云指数(Nebulas Rank,NR)\cite{Nabulasyellowpaper},我们提出的激励机制能够在一定程度上保证上述性质。

%4. 我们的算法

为了实现公平性,我们把用户调用DApp的次数作为影响评分的主要标准。正如各大排行网站出现的点击榜一样,这样能够保证好的DApp受到了更多关注,并能促进整个系统的活跃性。

为了抵抗女巫攻击,我们应用了星云指数(NR)本身具有的性质。我们让具有更高NR值的用户拥有更多的投票权。由于NR的设计保证了伪造高NR用户的困难性从而有效防止了用户这方面的女巫攻击。同时,我们最终奖励函数的凹函数性质能保证开发者将他的DApp拆分成多个小DApp不会提升他的收益。

为了防止收买,我们给与将票投给多个DApp的用户更高的投票权。因为被收买的用户往往只将票投给一个人,导致效用较低,这样就增加了收买的代价。

本紫皮书的结构如下:TBA
