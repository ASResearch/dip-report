\section{概要}

%1. 区块链关于DApp的介绍

一般而言,在传统的软件开发行业,开发者的获利方式主要包括,薪资、售卖自己主导或参与开发的软件以及在免费分发的软件中插入广告流量。

然而,平台方在软件的使用中所获得的利益并没有被公平的分配给相应的开发者。例如,某开发者使用Windows或iOS平台开发了某个被普遍使用的应用,开发者固然可以通过售卖或广告流量收费,
但是,用户为了使用该应用而购买Windows操作系统或搭载iOS系统的手机而付出的成本,并未公平的分配给相应的开发者。事实上,一个平台的广泛使用,离不开其上的优秀应用,
因此,平台方无视开发者利益的做法,一定程度侵害了开发者的利益。

在区块链中,去中心化应用(Decentralized Application, DApp)的开发者的利益同样被漠视。
2014年,以太坊提出在区块链上运行图灵完备的智能合约(Smart Contract),使得区块链从单纯的数字货币支付网络升级为了去中心化应用平台。
然而,DApp开发者的获利方式与传统的软件开发行业相比,并无明显区别,即,开发者并不能从去中心应用平台或区块链系统的发展中获利。

可以抽象的认为,区块链中出块奖励代表了区块链系统的增值,而出块奖励的分发决定了去中心化系统的激励方向。我们认为,区块链系统增加的价值本质上来源于新增的用户数据所蕴含的价值,这些新增的价值
应该公平的分发给为系统的新增价值做出贡献的各方,其中,就包括DApp开发者。然而,在以比特币为代表的区块链系统中,出块奖励被发放给了矿工节点;在诸多的基于PoS(Proof of Stake)的区块链系统中,
出块奖励被发放给了系统代币的持有者;同样的,在诸多的区块链系统中,DApp开发者的利益一定程度上都被漠视或侵害了。

一般而言,一个去中心化应用可以理解为为了实现特定功能的一系列智能合约的集合。
智能合约是一种旨在以信息化方式传播、验证或执行合同的计算机协议。智能合约允许在没有第三方的情况下进行可信交易。
从技术架构来看,大部分DApp通常以智能合约作为后端,同时采用了常见的前端技术与之交互,因此DApp形态既可以是传统PC客户端,也可以是移动App或者Web。

我们认为,去中心化应用平台、DApp开发者及DApp用户,这三者是互相促进的。
首先,去中心化应用平台的出现,扩大了区块链开发者这一群体,越来越多的开发者尝试开发DApp,成为DApp开发者,并从DApp的开发中获益;
其次,DApp开发者提供了丰富多样的DApp,扩大了区块链的应用场景,为区块链带来了更多的用户;
最后,DApp用户驱动着去中心化应用平台的不断的优化、升级,以满足更多的用户需求。
因此,更进一步的,我们认为,DApp开发者的利益应该得到公平的分配,并给予保证,这是区块链做为去中心化应用平台维持可持续发展的关键所在。

本文提出开发者激励协议(Developer Incentive Protocol,DIP),试图给予开发者激励,让DApp开发者能够公平的在去中心化应用平台的发展中获益。
我们认为开发者激励协议需要满足如下基本性质:
\begin{itemize}
	\item 公开性:链上的DApp激励协议与传统的评奖方式最大的不同在于,所有评分的机制必须是完全公开的,
	且其中任何统计、计算、评选的过程都是全程可见的。
	这样就杜绝了传统中心化评奖暗箱操作的可能。同时也不会出现票数统计出错等情况。
	最后,根据评选结果分配奖励的过程也会保证被执行,奖励分配正如链上交易一样可被追溯。
	\item 有效性:这也是任何评选机制所要满足的基本性质。我们期望DApp评分能够真实反映用户的评价,
	即排名高的DApp是活跃用户所喜欢的且经常被调用的,而评分低的DApp是用户鲜有问津的。
	\item 抗作弊:对于任何排名算法,都需要解决各类作弊问题。对于DIP而言,主要存在两类作弊问题。
	\begin{enumerate}
		\item 女巫攻击:区块链技术的一个重大特点就是一个用户建立新的节点地址代价是很小的。所以一个用户有可能建立多个由他控制的地址,并将他们伪装成多个正常用户来参与评选。
		一个好的激励协议应当保证每个用户无法通过女巫攻击带来巨大额外收益。
		\item 收买:由于我们衡量DApp好坏的主要指标是活跃用户调用的次数,一个DApp开发者有可能收买大量用户让他们调用自己的DApp以提高自己的排名从而获得更多奖励。
		这种作弊方式原则上无法杜绝,但我们期望激励协议能够让此类收买需要付出的代价变得很高以减少其出现的概率。
	\end{enumerate}
\end{itemize}

基于已有的星云指数(Nebulas Rank,NR)\cite{Nabulasyellowpaper},我们提出的激励机制能够在一定程度上保证上述性质。
\begin{itemize}

\item 为了实现公平性,我们把用户调用DApp的次数作为影响评分的主要标准。正如各大排行网站出现的点击榜一样,这样能够保证好的DApp受到了更多关注,并能促进整个系统的活跃性。

\item 为了抵抗女巫攻击,我们应用了星云指数(NR)本身具有的性质。我们让具有更高NR值的用户拥有更多的投票权。
由于NR的设计保证了伪造高NR用户的困难性从而有效防止了用户这方面的女巫攻击。同时,我们最终奖励函数的凹函数性质能保证开发者将他的DApp拆分成多个小DApp不会提升他的收益。

\item 为了防止收买,我们给与将票投给多个DApp的用户更高的投票权。因为被收买的用户往往只将票投给一个人,导致效用较低,这样就增加了收买的代价。
\end{itemize}

\begin{comment}
%4. 我们的算法



%相比于传统的应用,DApp具有如下特性:
%\begin{itemize}
%	\item 数据公开透明、不可篡改,应用数据存储在区块链上所有节点,而不依赖于任何中心化的服务器;
%	\item 应用逻辑公开、不可篡改,应用逻辑完全由智能合约代码决定,绝大多数智能合约会公布其源代码,从而保证可信度。
%\end{itemize}



\textcolor{red}{下面这个切入点太直白了,试试这个:区块链做为典型的社区驱动的项目,维护社区中各方的利益,是区块链平台不可推卸的责任和义务。
然而,目前,大多数公链平台漠视、甚至损坏开发者的利益,这对于整个区块链行业及生态的长远健康发展,是极为不利的。因此,本文提出开发者激励协议。出发点应该是我们要维护社区的利益,而不是为了让星云繁荣,提出了这个东西。所以后续的措辞可能都需要改一下}
对于区块链平台而言,DApp应用数量和质量直观反映了其社区繁荣程度,也影响着平台发展潜力。
迄今为止\textcolor{red}{注明时间},根据DAppradar统计以太坊上已有超过632款DApp\footnote{https://DAppradar.com/DApps.},相比之下目前大部分其他基础公链的DApp发行数量都远不及以太坊。值得注意的是,星云链一直致力于发展DApp生态并为开发者提供友好的开发平台。经过历时两个多月的激励计划\textcolor{red}{不要提激励计划},目前星云链上已有超过6871个DApp\footnote{https://incentive.nebulas.io/summary.html.} \textcolor{red}{这个统计和对比是不严谨的}。
%同时,星云和eos上也分别有xxx和xxx DApp。

伴随着DApp爆炸式增长,暴露出的问题是DApp的质量参差不齐。因此,
如何对海量的DApp进行评价以提供更好的用户体验,以及如何促进激励开发者开发新的高质量DApp成为每条公链必须解决的问题。

%2. 引入开发者激励协议的必要性(何谓激励?)

所谓激励是指通过特定的方法让用户产生行为动机。对应地,\textcolor{red}{星云开发者激励协议(DIP)是指以鼓励开发者开发优秀DApp为目的,以一套具有抗作弊性的DApp排名算法为核心,以对优秀DApp开发者给与NAS奖励为手段的一系列开源的激励机制的集合。}简单而言,优秀的DApp将在我们的排名算法中获得较高排名进而获得较高的奖励(在星云链上以官方代币NAS形式发放),以此来激励开发者。 我们期望,DIP能使每一个理性的诚实DApp开发者自发的设计优秀的DApp,正如日常生活常见的评奖活动一样。同时,每一个理性的非诚实的开发者无法通过各种作弊手段来骗取NAS奖励。




事实上,正如目前区块链中存在不可能三角(The Imposible Triangle)一样\footnote{https://news.8btc.com/the-impossible-triangle-of-blockchain.},设计一个能够满足上述特性的激励协议存在很大挑战,同时我们需要保证一定的高效性与可扩展性。
本紫皮书的结构如下:TBA
\end{comment}
